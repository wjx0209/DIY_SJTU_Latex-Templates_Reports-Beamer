%----------------------------------------------------------------------------------------
%    PACKAGES AND THEMES
%----------------------------------------------------------------------------------------

\documentclass[aspectratio=169,xcolor=dvipsnames]{beamer}
\usetheme{SimpleDarkBlue}

\usepackage{hyperref}
\usepackage{graphicx} % Allows including images
\usepackage{booktabs} % Allows the use of \toprule, \midrule and \bottomrule in tables

%----------------------------------------------------------------------------------------
%    TITLE PAGE
%----------------------------------------------------------------------------------------
\title{Experience Post}
\subtitle{How to Listen to a Classical Music Performance --- If You Are Not a Music Major}

\author{Jia-Xin Wang}

\institute
{
    \textit{China-UK Low Carbon College, 
    Shanghai Jiao Tong University,
    Shanghai, China}  % Your institution for the title page
}
\date{\today} % Date, can be changed to a custom date
\titlegraphic{\vspace{-0.35cm} \includegraphics[width=0.3\textwidth]{English logo.png}} % 替换为你的 logo
%----------------------------------------------------------------------------------------
%    PRESENTATION SLIDES
%----------------------------------------------------------------------------------------

\begin{document}

\begin{frame}
    % Print the title page as the first slide
    \titlepage
\end{frame}

\begin{frame}{Overview}
    % Throughout your presentation, if you choose to use \section{} and \subsection{} commands, these will automatically be printed on this slide as an overview of your presentation
    \tableofcontents
\end{frame}

%------------------------------------------------
\section{Background}
%------------------------------------------------

\begin{frame}{Background}
More and more young people who are not majoring in music choose to listen to symphonies, musicals, and operas.
    \begin{itemize}
        \item The number of people under 35 who listen to classical music has risen to 40\% (most of them listen to classical music on Bilibili and YouTube)
        \item We gradually realized that classical music is not only for those who have studied music systematically.
    \end{itemize}
    
\end{frame}

%------------------------------------------------

\begin{frame}{Blocks of Highlighted Text}
    In this slide, some important text will be \alert{highlighted} because it's important. Please, don't abuse it.

    \begin{block}{Block}
        Sample text
    \end{block}

    \begin{alertblock}{Alertblock}
        Sample text in red box
    \end{alertblock}

    \begin{examples}
        Sample text in green box. The title of the block is ``Examples".
    \end{examples}
\end{frame}

%------------------------------------------------

\begin{frame}{Multiple Columns}
    \begin{columns}[c] % The "c" option specifies centered vertical alignment while the "t" option is used for top vertical alignment

        \column{.45\textwidth} % Left column and width
        \textbf{Heading}
        \begin{enumerate}
            \item Statement
            \item Explanation
            \item Example
        \end{enumerate}

        \column{.45\textwidth} % Right column and width
        Lorem ipsum dolor sit amet, consectetur adipiscing elit. Integer lectus nisl, ultricies in feugiat rutrum, porttitor sit amet augue. Aliquam ut tortor mauris. Sed volutpat ante purus, quis accumsan dolor.

    \end{columns}
\end{frame}

%------------------------------------------------
\section{Second Section}
%------------------------------------------------

\begin{frame}{Table}
    \begin{table}
        \begin{tabular}{l l l}
            \toprule
            \textbf{Treatments} & \textbf{Response 1} & \textbf{Response 2} \\
            \midrule
            Treatment 1         & 0.0003262           & 0.562               \\
            Treatment 2         & 0.0015681           & 0.910               \\
            Treatment 3         & 0.0009271           & 0.296               \\
            \bottomrule
        \end{tabular}
        \caption{Table caption}
    \end{table}
\end{frame}

%------------------------------------------------

\begin{frame}{Theorem}
    \begin{theorem}[Mass--energy equivalence]
        $E = mc^2$
    \end{theorem}
\end{frame}

%------------------------------------------------

\begin{frame}{Figure}
    Uncomment the code on this slide to include your own image from the same directory as the template .TeX file.
    %\begin{figure}
    %\includegraphics[width=0.8\linewidth]{test}
    %\end{figure}
\end{frame}

%------------------------------------------------

\begin{frame}[fragile] % Need to use the fragile option when verbatim is used in the slide
    \frametitle{Citation}
    An example of the \verb|\cite| command to cite within the presentation:\\~

    This statement requires citation \cite{p1}.
\end{frame}

%------------------------------------------------

\begin{frame}{References}
    \footnotesize
    \bibliography{reference.bib}
    \bibliographystyle{apalike}
\end{frame}

%------------------------------------------------

\begin{frame}
    \Huge{\centerline{\textbf{The End}}}
\end{frame}

%----------------------------------------------------------------------------------------

\end{document}